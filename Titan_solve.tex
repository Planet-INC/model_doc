\section{Photochemistry solver}

\subsection{Equation}

The photochemical solving solves the diffusion and the chemistry, thus we solve:
\begin{equation}
\doverd{\conc_s}{t} = - \left(\doverd{\conc_s}{t}\right)_\text{diff} + \left(\doverd{\conc_s}{t}\right)_\text{kin}
\end{equation}
Which is expressed as:
\begin{equation}
\dot{\conc_s} = \omegadot_s - \gradien{z}{\transport_s} - \frac{2}{r}\transport_s
\label{Titan:photochem_eq}
\end{equation}
with \omegadot\ the contribution of the chemistry.
See appendix~\ref{math:photo_solve} for the details of the solving.

\subsection{Boundary conditions}

The lower bondary condition is given by the input parameters, the molar densities
are constrained. The upper boundary condition is given as a flux constrain. The
upper boundary is above the exobase, the limit above which we can consider that
the processes happening are collisionless. Thus the upper boundary condition
is expressed on the fluxes, stating that \doverdz{\conc} is the Jeans' flux
at the upper bound.

\section{Ionospheric solver}

The ionosphere is at steady-state with respect to the
neutral system. Thus we want to solve, neutral species
densities being constant:
\begin{equation}
\forall i \in \text{ ions system },\quad \doverdt{\conc_i} = 0
\label{ionosphere:steady_state}
\end{equation}
%
This is a somewhat ``easy'' solve, thus a Newton method is the
most appropriate, as it has the optimal convergence rate.
Newton's method is an iterative method, the idea is to start
at a first guess, solve for value of $x$ at which the tangent is
equal to zero, which will be the next guess (see Fig.~\ref{Newton:solve}).

The tangent at $x_n$ is an affine function, thus of the form
$ax + b$, with $a = f'(x_n) = \doverd{f}{x}(x_n)$. This function
admits the point $(x_n,f(x_n))$ as a solution, thus:
\begin{equation}
\begin{split}
              & f'(x_n) x_n + b = f(x_n) \\
\Rightarrow\; & b = f(x_n) - f'(x_n) x_n
\end{split}
\label{Newton:tangent}
\end{equation}
Thus, the tangent admits one root at:
\begin{equation}
\begin{split}
              & f'(x_n) x_\text{root} + f(x_n) - f'(x_n) x_n = 0 \\
\Rightarrow\; & f'(x_n) \left(x_\text{root} - x_n \right) + f(x_n) = 0
\end{split}
\label{Newton:tangent:root}
\end{equation}
the root being $x_{n+1}$.

Numerically, what happens is solving the equation
\begin{equation}
f'(x_n)X - f(x) = 0,
\label{Newton:solving}
\end{equation}
with $X = x_{n+1} - x_n$, then, trivially get $x_{n+1}$ from
$X$ and $x_n$.
\begin{figure}
\centering
\includegraphics{Newton}
\caption{\label{Newton:solve}From a first guess $x_0$, we iteratively
comes to the solution. The iteration is defined as $x_{n+1} = x_n - \frac{f(x_n)}{f'(x_n)}$.}
\end{figure}

In an atmospheric case, $x$ is the molar composition, thus the
vector of densities (\mat{\conc_0,\dots,\conc_N}), 
the function $f$ is the molar rate 
and the function $f'$ is the Jacobian matrix. Thus
the solving equation is:
\begin{equation}
\left[
\begin{array}{ccc}
\dOverd{\doverdt{\conc_0}}{\conc_0} & \dots  & \dOverd{\doverdt{\conc_0}}{\conc_N} \\
 \vdots                             & \ddots & \vdots \\
\dOverd{\doverdt{\conc_N}}{\conc_0} & \dots  & \dOverd{\doverdt{\conc_N}}{\conc_N} \\
\end{array}
\right] 
\left[X_0,\dots,X_n\right]^{\perp} - \left[\doverdt{\conc_0},\dots,\doverdt{\conc_N}\right] = 0
\label{Newton:matrix_solve}
\end{equation}

\Eigen\footnote{\EigenTux} solves the matrix system defined in Eq.~\ref{Newton:matrix_solve}, \PINC\ calculates
the needed values, \textit{i.e.} the jacobian matrix and mole sources vector.
