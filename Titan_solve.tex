\section{Photochemistry solver}

\subsection{Equation}

The photochemical solving solves the diffusion and the chemistry, thus we solve:
\begin{equation}
\doverd{\conc_s}{t} = - \left(\doverd{\conc_s}{t}\right)_\text{diff} + \left(\doverd{\conc_s}{t}\right)_\text{kin}
\end{equation}
and we develop thanks to equations~\ref{Titan:diffusion} and \ref{Titan:kinetics}
\begin{equation}
\begin{split}
\left(\doverd{\conc_s}{t}\right) & = \sum_{r\in \text{reactions}} \stoi{s,r}\rate_{r} \prod_{i\in\text{reactants}}\conc_i^{|\stoi{i,r}|} -
                                     \frac{1}{r^2}\doverd{\left(r^2\conc_s\omega_s\right)}{r} \\
                                 & = \sum_{r\in \text{reactions}} \stoi{s,r} \rate_r\prod_{i\in\text{reactants}}\conc_i^{|\stoi{i,r}|} -
                                     \frac{1}{r^2}\doverd{\left(r^2\right)}{r}\conc_s\omega_s - \doverd{\left(\conc_s\omega_s\right)}{r} \\
                                 & = \sum_{r\in \text{reactions}} \stoi{s,r} \rate_{r}\prod_{i\in\text{reactants}}\conc_i^{|\stoi{i,r}|} -
                                     \frac{2\conc_s\omega_s}{r} - \doverdz{\left(\conc_s\omega_s\right)} \\[\baselineskip]
                                 & = - \dOverdz{\conc_s\omega_s} - \frac{2\conc_s\omega_s}{r} + \sum_{r\in \text{reactions}} \stoi{s,r} \rate_{r}\prod_{i\in\text{reactants}}\conc_i^{|\stoi{i,r}|} 
\end{split}
\end{equation}
Finally:
\begin{equation}
\doverd{\conc_s}{t} = \omegadot_s - \dOverdz{\conc_s\omega_s} - \frac{2}{r}\conc_s\omega_s
\label{Titan:photochem_eq}
\end{equation}
with \omegadot\ the contribution of the chemistry.
See appendix~\ref{math:photo_solve} for the details of the solving.

\subsection{Boundary conditions}

The lower bondary condition is given by the input parameters, the molar densities
are constrained. The upper boundary condition is given as a flux constrain. The
upper boundary is above the exobase, the limit above which we can consider that
the processes happening are collisionless. Thus the upper boundary condition
is expressed on the fluxes, stating that \doverdz{\conc} is the Jeans' flux
at the upper bound.
