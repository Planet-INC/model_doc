\section{Generalities on chemistry}
Here is an overview of how the chemical kinetics are
handed. It is all taken
care of by \Antioch, see \GitAntioch\ for a detailed
description.

\subsection{Description of a chemical reaction}

A chemical reaction is characterized by a rate constant
\rate, which depends on the pressure and temperature.
A kinetics model is defined to represent the temperature
dependency while a chemical process codes for the pressure
dependency. As we treat only gaseous media and we make
the ideal gas approximation, pressure is concentration.

A kinetics model is called hereafter $\alpha$ and depends solely on
temperature. The kinetics models supported by \Antioch\ are given
in Tab.~\ref{Antioch_kinetics_models}
\begin{table}
\centering
\renewcommand{\arraystretch}{2}
\begin{tabular}{ll}\toprule
Kinetics model & Equation \\\midrule
Hercourt-Essen & $\alpha(T) = \preCoeff \left(\frac{T}{\Tref}\right)^\powerPar$ \\
Berthelot      & $\alpha(T) = \preCoeff \;\exp\!\left(D T\right)$ \\
Arrhenius      & $\alpha(T) = \preCoeff \;\exp\!\left(-\frac{\actEn}{\Runi T}\right)$ \\
Kooij          & $\alpha(T) = \preCoeff \left(\frac{T}{\Tref}\right)^\powerPar\exp\!\left(-\frac{\actEn}{\Runi T}\right)$ \\
Van't Hoff     & $\alpha(T) = \preCoeff \left(\frac{T}{\Tref}\right)^\powerPar\exp\!\left(-\frac{\actEn}{\Runi T} + D T\right)$ \\
\bottomrule
\end{tabular}
\caption{\label{Antioch_kinetics_models}Different kinetics models supported by \Antioch.}
\end{table}

A chemical process will then be a generalization of the kinetics model, adding
pressure dependency. Tab.~\ref{Antioch_chem_process} summarizes the supported
chemical process.
\begin{table}
\centering
\renewcommand{\arraystretch}{2}
\begin{tabular}{ll}\toprule
Chemical process & Equation \\\midrule
Elementary       & $\rate(T,\conc) = \alpha(T)$ \\
Duplicate        & $\rate(T,\conc) = \sum_{i=1}^n\alpha_i(T)$ \\
Three body       & $\rate(T,\conc) = \alpha(T)\left(\sum_s \epsilon_s\conc_s\right)$ \\
Falloff          & $\rate(T,\conc) = \alpha_\infty(T)\frac{\conc\,\alpha_0(T)}{\conc\,\alpha_0(T) + \alpha_\infty(T)} F$ \\\cmidrule{2-2}
                 &\begin{tabular}{>{\small}ll} 
                   Lindemann & $F = 1$ \\
                   Troe      & $F =$ a complicated equation
                  \end{tabular}\\
\bottomrule
\end{tabular}
\caption{\label{Antioch_chem_process}Different chemical processes supported by \Antioch.}
\end{table}


\section{Photochemistry}
\subsection{Reaction}
A photochemical reaction typically is:
\begin{chemequation}
\renewcommand{\arraystretch}{1.2}
\ce{CH4 ->[\sigma(\lambda)][\fhv(\lambda)]}\left\{\begin{array}{l}
                    \ce{-> ^1CH2 + H2} \\
                    \ce{-> CH3 + H}\\
                    \ce{-> CH2 + 2 H}\\
                    \ce{-> CH4+}\\
                    \ce{-> CH3+ + H}\\
                    \ce{-> CH2+ + H2}\\
                    \ce{-> CH+ + H2 + H}\\
                    \ce{-> H+ + CH3}\\
                    \ce{-> CH + H2 + H}\\
                    \end{array}\right.
\end{chemequation}
The rate constant is computed through the equation
\begin{equation}
k(z) = \int_0^\infty \cs(\lambda)\fhv(z,\lambda)\dd\lambda
\end{equation}
with $\fhv(z,\lambda)$ the photon flux at the considered altitude.
Thus the calculations is made in three steps:
\begin{enumerate}
\item calculate the function $\fhv(z,\lambda)$,
\item harmonize the $\lambda$ grid,
\item integrate.
\end{enumerate}

\subsection{Photon flux at altitude \texorpdfstring{$z$}{z}}

\begin{equation}
\fhv(z,\lambda) = \fhvtop(\lambda) \exp\left(-\tau(z,\lambda)\right)
\end{equation}
with \fhvtop\ the photon flux at the top of the atmosphere, and
\begin{equation}
\tau(z,\lambda) = \chapman\sum_s\sigma_s(\lambda)\int_z^\infty n_s(z')\dd z'
\end{equation}
with $n_s(z')$ the molar density of species $s$ at altitude $z'$.
the Chapman function depends on the
incident angle $\chi$.\\
For $\chi \le 75^\circ$
\begin{equation}
\chapman =  \frac{1}{\cos(\chi)},
\end{equation}
for $75^\circ < \chi \le 90^\circ$
\begin{equation}
\chapman = \sqrt{\frac{\pi \x}{2}} 
                \left(1 - \erf\left(\sqrt{\frac{\x}{2}}|\cos(\chi)|\right)\right)\exp\left(\frac{\x}{2}\cos^2(\chi)\right),
\end{equation}
for $90^\circ < \chi$
\begin{equation}
\begin{split}
\chapman = & \sqrt{2\pi \x}\bigg[\sqrt{\sin(\chi)}\exp\left(\x\left(1-\sin(\chi)\right)\right) \\
           &  -\frac{1}{2}\exp\left(\frac{\x}{2}\cos^2(\chi)\left(1-\erf\left(\sqrt{\frac{\x}{2}}|\cos(\chi)|\right)\right)\right)\bigg]
\end{split}
\end{equation}
And the definition of \x\ is given by
\begin{equation}
\x(z) = \frac{\RTitan + z}{\sch(z)}
\end{equation}
with $\sch(z)$ being the scale height of definition
\begin{equation}
\sch(z) = \frac{\kb T(z)}{\gloc(z)\mean{\mass}}
\end{equation}
with \kb\ the Boltzmann constant, 
\mean{\mass}\ the atmospheric mass ($\mean{\mass} = \frac{\Mm(z)}{\Nav}\conc_\text{tot}(z)$, 
$\Mm(z)$ the molar mass of the atmosphere at altitude $z$,
\Nav\ the Avogadro number, $\conc_\text{tot}(z)$ the density at altitude $z$), 
$\gloc(z)$ the gravitationnal field at altitude $z$, given by the equation
\begin{equation}
\gloc(z) = \frac{\Guni\cdot\MTitan}{\left(\RTitan + z\right)^2}
\label{gfield}
\end{equation}

\subsection{Photon flux at the top of the atmosphere}
\begin{equation}
\fhvtop(\lambda) = \frac{\fhvAU}{\dSS^2}
\end{equation}
with \dSS\ the distance between the Sun and Saturn.

\subsection{Finally}

The whole computations starts with the inputs:
\Guni, the universal gravitationnal constant,
\MTitan, the mass of Titan,
\RTitan, the radius of Titan,
\Nav, the Avogadro number,
\kb, the Boltzmann constant,
$\{n_s\}$, the molar densities of the species at each altitude,
\fhvAU, the solar flux at 1 AU of the Sun,
\dSS, the distance Sun-Saturn,
$\chi$, the zenith angle,
$\{\cs_s(\lambda)\}$, the cross-section of the species.
\begin{center}
\begin{tabular}{rrrrrrrr}
\Guni   & \rdelim\}{3}{28pt}[\gloc(z)] & & & & & & \\
\MTitan &      & \rdelim\}{7}{35pt}[\sch(z)] & & & & &  \\
\RTitan &      & &                       &    & & & \\
        & \Nav & \RTitan & \rdelim\}{2}{27pt}[\x(z)]&\rdelim\}{3}{58pt}[\chapman] & & & \\[5pt]
        & \kb  & & $\chi$                &             & \rdelim\}{4}{39pt}[$\tau(z,\lambda)$]&\\
        & $T(z)$ & &                     & \sch(z)     & & \rdelim\}{5}{50pt}[$\fhv(z,\lambda)$]\\
\multirow{2}{*}{$\{n_s\}$} 
        & \rdelim\}{2}{28pt}[\Mm] & &   & $\{n_s\}$   & &\\
        &      & &                       & $\{\cs_s\}$ & &\\[5pt]
        &      & &                       & $\fhvAU(\lambda)$ & \rdelim\}{2}{39pt}[$\fhvtop(\lambda)$] &\\
        &      & &                       & \dSS        & &
\end{tabular}
\end{center}


\begin{figure}
\centering
\includegraphics[width=\textwidth]{sunray}
\caption{\label{sunray}Titan under the sun (from \citet[Fig.~3.3]{Haye2005})}
\end{figure}

\subsection{The data}

Starting with $\fhvAU(\lambda)$, thus values of
irradiance on a grid of wavelengthes. We calculate
$\fhv(z,\lambda)$ on the same grid with the given
workflow. Then we need to integrate $\cs(\lambda)\fhv(z,\lambda)$
for every altitude.

\begin{figure}
\centering
\includegraphics{cs_CH4}
\caption{\label{csCH4}Cross-section of \ce{CH4}}
\end{figure}



\section{Electrochemistry}
\subsection{The reaction}
\begin{chemequation}
\renewcommand{\arraystretch}{1.5}
\ce{CH4 ->[\sigma(E)][\fe(E)]}\left\{\begin{array}{l}
                            \ce{-> C+ + 2\, H2} \\
                            \ce{-> C+ + 2\, H + H2} \\
                            \ce{-> C+ + 4\, H} \\
                            \ce{-> CH+ + H2 + H} \\
                            \ce{-> CH+ + 3\, H} \\
                            \ce{-> CH2+ + H2} \\
                            \ce{-> CH2+ + 2\, H} \\
                            \ce{-> CH3+ + H} \\
                            \ce{-> CH4+}
                           \end{array}\right.
\end{chemequation}
This is similar to the photochemistry,
given an electronic flux profile, we compute the
rate constant with the equation
\begin{equation}
\rate(z) = \int_0^\infty \cs(E)\fe(z,E)\dd E
\end{equation}


\section{Bimolecular reactions}
\label{Titan_bimol_sec}
\subsection{The reaction}
\begin{chemequation}
\ce{H + CH -> H2 + C}
\end{chemequation}

These reactions are all considered elementary processes, and
the kinetics model used is Hercourt-Essen, Arrhenuis or Kooij
(see Tab.~\ref{Antioch_kinetics_models}).


\section{Falloff reactions}
\subsection{The reaction}
\begin{chemequation}
\ce{C + H2 -> ^3CH2}
\end{chemequation}
These reactions are typically modeled by Lindemann falloff,
the kinetics model is the same as the bimolecular reactions
(section~\ref{Titan_bimol_sec}), we obtain thus as rate constant:
\begin{equation}
k(T,\concAtm) = \frac{\concAtm \alpha_0(T) \alpha_\infty(T)}{\alpha_\infty(T) + \concAtm \alpha_0(T)}
\end{equation}
This type of reaction depends on the pressure (\concAtm),
$\alpha_0(T)$ is the kinetics model used at low pressure,
$\alpha_\infty(T)$ at high pressures.


\section{Dissociative electronic impact reactions}
\subsection{The reaction}
\begin{chemequation}
\renewcommand{\arraystretch}{1.5}
\ce{CH3+ + e- ->} \left\{\begin{array}{l}
                        \ce{->}\left\{\begin{array}{l}
                                      \ce{-> ^1CH2 + H} \\
                                      \ce{-> ^3CH2 + H} \\
                                      \end{array}\right. \\
                        \ce{-> CH + H2} \\
                        \ce{-> CH + 2 H} \\
                        \ce{-> C + H2 + H} \\
                         \end{array}\right.
\end{chemequation}
These reactions are coded by an elemetary process with a Hercourt-Essen
kinetics model:
\begin{equation}
k(T) = \preCoeff \left(\frac{T}{\Tref}\right)^\powerPar
\end{equation}


\section{Kinetics computations}
The kinetics law is given by
\begin{equation}
\left(\doverd{\conc_s}{t}\right)_{\text{kin},r} = \stoi{s,r}\; k_{r} \prod_{i\in\text{reactants}}\conc_i^\order{i}
\end{equation}
for the contribution of reaction $r$ to a species $s$ concentration's
time derivative. \stoi{s,r}\ is the stoichiometric coefficient of species
$s$ in reaction $r$. The elementary process enables to make the hypothesis
\begin{equation}
\forall r \in \text{reactions}\qquad
\order{i} = |\stoi{r,i}|
\end{equation}

The total kinetics contribution for species $s$ would then be:
\begin{equation}
\left(\doverd{\conc_s}{t}\right)_\text{kin} = \sum_{r\in \text{reactions}} \stoi{s,r} k_{r} \prod_{i\in\text{reactants}}\conc_i^{|\stoi{i,r}|}
\label{Titan:kinetics}
\end{equation}
