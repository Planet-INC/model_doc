An atmospheric molecule, and by extension, an atmosphere, is characterized by
its scale height, i.e. the distance over which the density is divided by $e$, and
its mean free path, i.e. the mean length a molecule can cover before undergoing
a collision.
\begin{equation}
\begin{split}
\sch(z)  &    = \frac{\kb T(z)}{\gloc(z)\mass} \\[5pt]
\uu [km] &\uu = 10^{-3} \frac{[\ukb] [K]}{[m\,s^{-2}][kg]}
\end{split}
\label{scale_height}
\end{equation}
with \kb\ the Boltzmann constant, 
\mass\ the molecule mass ($\mass = \frac{\Mm}{\Nav}$, with
\Mm\ the molar mass and \Nav\ the Avogadro number), 
$\gloc(z)$ the gravitationnal field at altitude $z$, given by the equation
\begin{equation}
\begin{split}
\gloc(z)        &     = \frac{\Guni\cdot\MPlanet}{\left(\RPlanet + z\right)^2} \\[5pt]
\uu [m\,s^{-2}] & \uu = 10^{-6} \frac{[\uGuni] [kg]}{\left([km] + [km]\right)^2}
\end{split}
\label{gfield}
\end{equation}
%
The atmospheric scale height is thus expressed with the mean atmospheric
mass \mean{m}, given by:
\begin{equation}
\begin{split}
\mean{m} & = \sum_s \mass_s \frac{\conc_s}{\concAtm} \\[5pt]
\uu [kg] & \uu = \sum_s [kg] \frac{[cm^{-3}]}{[cm^{-3}]}
\end{split}
\end{equation}
%
\begin{equation}
\begin{split}
\doverd{\mean{m}}{\conc_k} & = \frac{\mass_s}{\concAtm} - \frac{\mean{m}}{\concAtm}
\\[5pt]
\uu \frac{[kg]}{[cm^{-3}]} & \uu = \frac{[kg]}{[cm^{-3}]} - \frac{[kg]}{[cm^{-3}]}
\end{split}
\end{equation}
%
Giving therefore:
\begin{equation}
\begin{split}
\doverd{\sch_a}{\conc_k}   &     =  \sch_a\left(\frac{1}{\concAtm} - \frac{\mass_s}{\sum_i\mass_i\conc_i}\right) \\[5pt]
\uu \frac{[km]}{[cm^{-3}]} & \uu =  [km]\left(\frac{[-]}{[cm^{-3}]} - \frac{[kg]}{\sum_i[kg][cm^{-3}]}\right)
\end{split}
\end{equation}

The mean free path is a collision process. Thus the molecule is approximated
by a hard sphere (rigid molecule model), and its hard sphere radius enables
to model a collision cross-section between two molecules $i$ and $j$:
\begin{equation}
\begin{split}
\sigma^{(ij)}_\text{col} & = \pi\left(r_i + r_j\right)^2 \\
\uu [cm^2]               & \uu = \left([cm] + [cm]\right)^2
\end{split}
\label{sigma_collision}
\end{equation}
The mean free path of species $s$ is then expressed as:
\begin{equation}
\begin{split}
\mfp_s   &     = \frac{1}{\sum_{i=1}^{N}\conc_i\sigma^{(si)}_\text{col}\sqrt{1 + \frac{\mass_s}{\mass_i}}} \\
\uu [km] & \uu = 10^{-5}\left([cm^{-3}] [cm^2] \left([-] + [kg][kg]^{-1}\right)^{\frac{1}{2}}\right)^{-1}
\end{split}
\label{mean_free_path}
\end{equation}
The exobase is the altitude at wich the mean free path is equal to the scale height. At this
altitude, collisions can be neglected.
