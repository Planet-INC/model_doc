\subsection{The equation}

Let's consider an equation to solve:
\begin{equation}
\begin{split}
& \ddoverd{u}{x} = 0,\qquad x \in [0,1] \\
& u(0) = 0, \; u(1) = 0
\end{split}
\label{sec:math:eq}
\end{equation}
This is a simple equation to solve.
We multiply by a function $v$ from a space $\mathcal{V}$
(see~\ref{sec:math:ipp} for integration by parts):
\begin{equation}
\begin{split}
            & \ddoverd{u}{x} = 0 \\
\Rightarrow & \ddoverd{u}{x} v = 0 \\
\Rightarrow & \int_0^1\ddoverd{u}{x} v \dd x = 0 \\
\Rightarrow & -\int_0^1\doverd{u}{x}\doverd{v}{x} \dd x + \left[\doverd{u}{x}v\right]_0^1 = 0 \qquad \forall v \in \mathcal{H}^1\\
\end{split}
\end{equation}
$\mathcal{H}^1$ is the ensemble of functions with at least one square-integrable
first derivative.
There is a choice in the $v$ function, and in this case, as we know the solutions
at the boundary conditions, we can get rid of them with a function $v$ such as
$v(0) = v(1) = 0$. We obtain thus
\begin{equation}
\int_0^1\doverd{u}{x}\doverd{v}{x} \dd x = 0, \qquad \forall v \in \mathcal{H}_0^1
\label{sec:math:weak_form}
\end{equation}
with $\mathcal{H}_0^1$ the ensemble of the functions having a finite countable numbers
of differentiable points and null at the points $0$ and $1$.

Now we need to discretize it. We consider a mesh of $N$ points between $0$ and $1$, we
express then:
\begin{equation}
\begin{split}
& u(x) = \sum_{i=1}^{N} \ui_i \phi_i \\
& v(x) = \sum_{i=1}^{N} \vi_i \phi_i
\end{split}
\label{sec:math:discretization}
\end{equation}
with $\{\phi\}$ the basis functions.

\subsection{Description of \texorpdfstring{$\phi$}{the basis functions}}

We need any function differentiable in any countable number of points such as
it is null at $0$ and $1$. A nice
choice is a hat function:
\begin{equation}
\left\{\begin{array}{l@{,\qquad}r}
\phi_i(x) = \frac{1}{x_i - x_{i-1}} - \frac{x_{i-1}}{x_i - x_{i-1}} & x \in [x_{i-1},x_i] \\
\phi_i(x) = \frac{1}{x_i - x_{i+1}} - \frac{x_{i+1}}{x_i - x_{i+1}} & x \in ]x_i,x_{i+1}] \\
\phi_i(x) = 0                                                       & \text{otherwise}
\end{array}\right.
\end{equation}
\begin{figure}
\centering
\includegraphics{hat_function}
\caption{\label{sec:math:hat_function}A hat function: here is $\phi_i$.}
\end{figure}

\subsection{Discretization and solving}

Using the set of basis functions (Eq.~\ref{sec:math:discretization}), we rewrite 
Eq.~\ref{sec:math:weak_form}.
\begin{equation}
\begin{split}
\int_0^1\sum_{i=1}^N\ui_i\doverd{\phi_i}{x}\sum_{j=1}^N\vi_j\doverd{\phi_j}{x} \dd x & = 0 \\
\sum_{i=1}^N\sum_{j=1}^N\ui_i\vi_j\int_0^1\doverd{\phi_i}{x}\doverd{\phi_j}{x} \dd x & = 0 \\
\end{split}
\label{sec:math:weak_discrete}
\end{equation}
Now if you look at the functions $\phi$, you'll see that the integrals are null except
around the considered node:
\begin{equation}
\int_0^1\doverd{\phi_i}{x} \dd x = \int_{x_{i-1}}^{x_{i+1}}\doverd{\phi_i}{x} \dd x
\end{equation}
So basically, you can define a nice matrix:
\begin{equation}
\matrice{K},\; K_{ij} = \int_{x_{i-1}}^{x_{i+1}}\doverd{\phi_i}{x}\doverd{\phi_j}{x} \dd x
\end{equation}
and using the $v$ function such as $\forall i \in [1,N], \; \vi_i = 1$, noting
\matrice{u} the $\ui_i$ column vector, we can
rewrite~\ref{sec:math:weak_discrete}:
\begin{equation}
\matrice{K} \matrice{u} = \matrice{b}
\end{equation}
with $\matrice{b} = 0$. So now it all comes down to a linear solve, and that's what
\LibMesh\ does nicely. \GRINS\ provides the formulation.

\subsection{Some definitions}
To better cope with \GRINS\ and \LibMesh, let's give some definitions here:
\\\\
\begin{tabular}{ll}\toprule
$\doverd{\phi_i}{x}\doverd{\phi_j}{x}$ & \matrice{K}, stiffness matrix\\
$\phi_i\phi_j$                         & mass term\\
\bottomrule
\end{tabular}

%%%% local defs
\newcommand{\intvol}  {\ensuremath{\int_{\text{atm}}}}
\newcommand{\intr}    {\ensuremath{\int_{\varphi_0}^{\varphi_1}\int_{\theta_0}^{\theta_1}\int_{r_0}^{r_1}}}
\newcommand{\dVs}     {\ensuremath{r^2\sin(\theta)\dd\theta\dd\varphi\dd r}}
\newcommand{\intboundary}{\ensuremath{\int_{\mathcal{S}_\text{atm}}}}
\newcommand{\dboundary}  {\ensuremath{{\dd\mathcal{S}_\text{atm}}}}

\subsection{Photochemical equation}
\label{math:photo_solve}

The photochemical equation, Eq.~\ref{Titan:photochem_eq}, is given by:
\begin{equation}
\doverd{u}{t} - \chi + \diverge{\transport} = 0,
\end{equation}
we will thus operate the integration by parts on the \diverge{\transport} term.

\subsubsection{Spherical coordinates}

The divergence in spherical coordinates $(r, \theta, \varphi)$ is given by:
\begin{equation}
\diverge{\cdot} = \frac{1}{r^2}\dOverd{r^2\cdot}{r} + 
                  \frac{1}{r\sin(\theta)}\dOverd{\sin(\theta)\cdot}{\theta} +
                  \frac{1}{r\sin(\theta)}\doverd{\cdot}{\varphi}.
\end{equation}
%
We can rewrite the equation in spherical coordinates:
\begin{equation}
\begin{split}
\dot{\conc_s} & = \chi -  \frac{1}{r^2}\Gradien{r}{r^2\transport} - 
                          \frac{1}{r\sin(\theta)}\Gradien{\theta}{\sin(\theta)\transport} -
                          \frac{1}{r\sin(\theta)}\gradien{\varphi}{\transport} \\[5pt]
              & = \chi - \frac{r^2}{r^2}\gradien{r}{\transport} - \frac{\transport}{r^2}\Gradien{r}{r^2} -
                         \frac{\sin(\theta)}{r\sin(\theta)}\gradien{\theta}{\transport} -
                         \frac{\transport}{r\sin(\theta)}\gradien{\theta}{\sin(\theta)} -
                         \frac{1}{r\sin(\theta)}\gradien{\varphi}{\transport} \\[5pt]
              & = \chi - \gradien{r}{\transport} - \frac{2}{r}\transport -
                         \frac{1}{r}\gradien{\theta}{\transport} -
                         \frac{\cos(\theta)}{r\sin(\theta)}\transport -
                         \frac{1}{r\sin(\theta)}\gradien{\varphi}{\transport}
\end{split}
\end{equation}
%
We solve the steady state, therefore $\dot{\conc_s}=0$, we integrate on the atmosphere:
\begin{equation}
\int_\Omega \left( \chi - \gradien{r}{\transport} - \frac{2}{r}\transport -
                         \frac{1}{r}\gradien{\theta}{\transport} -
                         \frac{\cos(\theta)}{r\sin(\theta)}\transport -
                         \frac{1}{r\sin(\theta)}\gradien{\varphi}{\transport} \right) \dd\Omega
 = 0,
\end{equation}
the volume element $\dd\Omega$ is expressed in spherical coordinates:
\begin{equation}
\int_{\varphi = 0}^{2\pi}\int_{\theta = 0}^{\pi}\int_{r_0}^{r_1}
\left( \chi - \gradien{r}{\transport} - \frac{2}{r}\transport -
                         \frac{1}{r}\gradien{\theta}{\transport} -
                         \frac{\cos(\theta)}{r\sin(\theta)}\transport -
                         \frac{1}{r\sin(\theta)}\gradien{\varphi}{\transport} \right) 
r^2\sin(\theta)\dd\varphi\dd\theta\dd r
= 0.
\end{equation}
%
As of now, \transport\ depends only on $r$, giving thus 
$\gradien{\theta}{\transport} = \gradien{\varphi}{\transport} = 0$.
%
\begin{equation}
\begin{split}
\int_{\varphi = 0}^{2\pi}\int_{\theta = 0}^{\pi}\int_{r_0}^{r_1}
\left( \chi - \gradien{r}{\transport} - \frac{2}{r}\transport - \frac{\cos(\theta)}{r\sin(\theta)}\transport 
\right) 
r^2\sin(\theta)\dd\varphi\dd\theta\dd r
 & = 0 \\[5pt]
2\pi\int_{r_0}^{r_1}
\left(
\chi - \gradien{r}{\transport} - \frac{2}{r}\transport
\right)
r^2
\int_{0}^{\pi}\sin(\theta)\dd\theta\dd r
 - 
2\pi\int_{r_0}^{r_1}\int_{0}^{\pi}
\left(
\frac{\cos(\theta)}{r\sin(\theta)}\transport  \right)
r^2\sin(\theta)\dd\theta\dd r
 & = 0 \\[5pt]
4\pi\int_{r_0}^{r_1}\left(
\chi - \gradien{r}{\transport} - \frac{2}{r}\transport
\right)
r^2\dd r
-
2\pi\int_{r_0}^{r_1} r\transport \int_{0}^{\pi}
\cos(\theta)\dd\theta\dd r
 & = 0
\end{split}
\end{equation}
%
Finally, the equation is:
\begin{equation}
\int_{r_0}^{r_1}\left(\chi - \gradien{r}{\transport} - \frac{2}{r}\transport\right)r^2\dd r = 0
\label{solve:photochemical_equation}
\end{equation}

\subsubsection{The weak form}

For any function $v \in \mathcal{H}^1$\footnote{$v$ is called the \emph{test function}}:
\begin{equation}
\int_{r_0}^{r_1}\left(\chi v - \gradien{r}{\transport} v - \frac{2}{r}\transport v \right)r^2\dd r = 0
\label{solve:weak_form_start}
\end{equation}
%
integration by parts provides:
\begin{equation}
\int_{r_0}^{r_1}\gradien{r}{\transport} v r^2 \dd r = - \int_{r_0}^{r^1} \transport \left(r^2\gradien{r}{v} + 2 r v\right)\dd r
                                                    + \left[\transport v r^2\right]_{r_0}^{r_1}
\end{equation}
%
substitution in Eq.~\ref{solve:weak_form_start} provides:
\begin{equation}
\int_{r_0}^{r_1}\left(\chi v + \transport\gradien{r}{v} + \transport\frac{2}{r} v  - \frac{2}{r}\transport v \right)r^2\dd r
 + \left[\transport r^2 v\right]_{r_0}^{r^1} = 0.
\end{equation}
%
As this holds for any $v \in \mathcal{H}^1$, in particular for $v$ such as
\[
\left\{\begin{array}{l}
         v(r_0) = 0 \\
         v(r_1) = 0 \\
        \end{array}\right.,
\]
%
we have finally the weak form of equation \ref{solve:photochemical_equation}:
\begin{equation}
\int_{r_0}^{r_1}\left(\chi v + \transport\gradien{r}{v} \right)r^2\dd r
\end{equation}

\subsubsection{Residual}

Using expression \ref{Titan:transport_expression} for the transport, we can
express the residual for any species $s$:
\begin{equationCode}{PlanetPhysics}
F_s = \left(\chi_s v + \omega_{As}\gradien{r}{\conc_s}\gradien{r}{v} + \omega_{Bs}\conc_s\gradien{r}{v} \right)r^2
\end{equationCode}

\subsubsection{The jacobian}

We discretize the test function $v$ and the solutions $\conc$ on
a basis of function $\phi$ such as:
\[
  \left\{\begin{array}{l}
         \conc = \displaystyle\sum_i^N \ui_i\phi_i \\[5pt]
         v     = \displaystyle\sum_i^N \vi_i\phi_i \\
         \end{array}\right.
\]
%
The jacobian will be defined by:
\[
J_{i,j} = \doverd{F(i)}{\phi_j} = 
        \left(\dOverd{\chi_s \phi_i + \omega_{As}\gradien{r}{\conc_s}\gradien{r}{\phi_i} + \omega_{Bs}\conc_s\gradien{r}{\phi_i}}{\phi_j} \right) \vi_i r^2 
\]
%
To performs this, for every species $s$, we consider all the contributions
from all species $t$. We have as the jacobian expression:
\begin{equationCode}{PlanetPhysics}
\begin{split}
J_{i,j} & = \sum_t\Bigg(\doverd{\chi_s}{\conc_t} \phi_i \phi_j +  \\  
        & \hphantom{= \sum_t\Bigg(}
            \doverd{\omega_{As}}{\conc_t}\gradien{r}{\conc_s}\gradien{r}{\phi_i}\phi_j + 
            \omega_{As}\gradien{r}{\phi_i}\gradien{r}{\phi_j}  + \\ 
        & \hphantom{= \sum_t\Bigg(}
             \doverd{\omega_{Bs}}{\conc_t}\conc_s\gradien{r}{\phi_i}\phi_j +
            \omega_{Bs}\gradien{r}{\phi_i}\phi_j  \\
        & \hphantom{= \sum_t\Bigg(} \Bigg)\vi_i r^2 
\end{split}
\end{equationCode}

At this point, we link to \GRINS\footnote{\GitGrins}: the planetary code provides $\chi$,
$\omega_A$, $\omega_B$ and the required derivatives. \GRINS\ and \LibMesh\footnote{\GitLibmesh} (the underlying library)
provide the $\phi$ fonctions, \conc, their derivatives and solve.
