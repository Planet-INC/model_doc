\subsection{The equation}

Let's consider an equation to solve:
\begin{equation}
\begin{split}
& \ddoverd{u}{x} = 0,\qquad x \in [0,1] \\
& u(0) = 0, \; u(1) = 0
\end{split}
\label{sec:math:eq}
\end{equation}
This is a simple equation to solve.
We multiply by a function $v$ from a space $\mathcal{V}$
(see~\ref{sec:math:ipp} for integration by parts):
\begin{equation}
\begin{split}
            & \ddoverd{u}{x} = 0 \\
\Rightarrow & \ddoverd{u}{x} v = 0 \\
\Rightarrow & \int_0^1\ddoverd{u}{x} v \dd x = 0 \\
\Rightarrow & -\int_0^1\doverd{u}{x}\doverd{v}{x} \dd x + \left[\doverd{u}{x}v\right]_0^1 = 0 \qquad \forall v \in \mathcal{H}^1\\
\end{split}
\end{equation}
$\mathcal{H}^1$ is the ensemble of functions with at least one square-integrable
first derivative.
There is a choice in the $v$ function, and in this case, as we know the solutions
at the boundary conditions, we can get rid of them with a function $v$ such as
$v(0) = v(1) = 0$. We obtain thus
\begin{equation}
\int_0^1\doverd{u}{x}\doverd{v}{x} \dd x = 0, \qquad \forall v \in \mathcal{H}_0^1
\label{sec:math:weak_form}
\end{equation}
with $\mathcal{H}_0^1$ the ensemble of the functions having a finite countable numbers
of differentiable points and null at the points $0$ and $1$.

Now we need to discretize it. We consider a mesh of $N$ points between $0$ and $1$, we
express then:
\begin{equation}
\begin{split}
& u(x) = \sum_{i=1}^{N} \ui_i \phi_i \\
& v(x) = \sum_{i=1}^{N} \vi_i \phi_i
\end{split}
\label{sec:math:discretization}
\end{equation}
with $\{\phi\}$ the basis functions.

\subsection{Description of \texorpdfstring{$\phi$}{the basis functions}}

We need any function differentiable in any countable number of points such as
it is null at $0$ and $1$. A nice
choice is a hat function:
\begin{equation}
\left\{\begin{array}{l@{,\qquad}r}
\phi_i(x) = \frac{1}{x_i - x_{i-1}} - \frac{x_{i-1}}{x_i - x_{i-1}} & x \in [x_{i-1},x_i] \\
\phi_i(x) = \frac{1}{x_i - x_{i+1}} - \frac{x_{i+1}}{x_i - x_{i+1}} & x \in ]x_i,x_{i+1}] \\
\phi_i(x) = 0                                                       & \text{otherwise}
\end{array}\right.
\end{equation}
\begin{figure}
\centering
\includegraphics{hat_function}
\caption{\label{sec:math:hat_function}A hat function: here is $\phi_i$.}
\end{figure}

\subsection{Discretization and solving}

Using the set of basis functions (Eq.~\ref{sec:math:discretization}), we rewrite 
Eq.~\ref{sec:math:weak_form}.
\begin{equation}
\begin{split}
\int_0^1\sum_{i=1}^N\ui_i\doverd{\phi_i}{x}\sum_{j=1}^N\vi_j\doverd{\phi_j}{x} \dd x & = 0 \\
\sum_{i=1}^N\sum_{j=1}^N\ui_i\vi_j\int_0^1\doverd{\phi_i}{x}\doverd{\phi_j}{x} \dd x & = 0 \\
\end{split}
\label{sec:math:weak_discrete}
\end{equation}
Now if you look at the functions $\phi$, you'll see that the integrals are null except
around the considered node:
\begin{equation}
\int_0^1\doverd{\phi_i}{x} \dd x = \int_{x_{i-1}}^{x_{i+1}}\doverd{\phi_i}{x} \dd x
\end{equation}
So basically, you can define a nice matrix:
\begin{equation}
\matrice{K},\; K_{ij} = \int_{x_{i-1}}^{x_{i+1}}\doverd{\phi_i}{x}\doverd{\phi_j}{x} \dd x
\end{equation}
and using the $v$ function such as $\forall i \in [1,N], \; \vi_i = 1$, noting
\matrice{u} the $\ui_i$ column vector, we can
rewrite~\ref{sec:math:weak_discrete}:
\begin{equation}
\matrice{K} \matrice{u} = \matrice{b}
\end{equation}
with $\matrice{b} = 0$. So now it all comes down to a linear solve, and that's what
\LibMesh\ does nicely. \GRINS\ provides the formulation.

\subsection{Some definitions}
To better cope with \GRINS\ and \LibMesh, let's give some definitions here:
\\\\
\begin{tabular}{ll}\toprule
$\doverd{\phi_i}{x}\doverd{\phi_j}{x}$ & \matrice{K}, stiffness matrix\\
$\phi_i\phi_j$                         & mass term\\
\bottomrule
\end{tabular}

%%%% local defs
\newcommand{\intvol}  {\ensuremath{\int_{\text{atm}}}}
\newcommand{\intr}    {\ensuremath{\int_{\varphi_0}^{\varphi_1}\int_{\theta_0}^{\theta_1}\int_{r_0}^{r_1}}}
\newcommand{\dVs}     {\ensuremath{r^2\sin(\theta)\dd\theta\dd\varphi\dd r}}
\newcommand{\intboundary}{\ensuremath{\int_{\mathcal{S}_\text{atm}}}}
\newcommand{\dboundary}  {\ensuremath{{\dd\mathcal{S}_\text{atm}}}}

\subsection{Photochemical equation}
\label{math:photo_solve}
The photochemical equation, Eq.~\ref{Titan:photochem_eq}, can be written as:
\begin{equation}
\doverd{u}{t} - \chi + \diverge{\transport} = 0
\end{equation}
we will thus operate the integration by parts on the \diverge{\transport} term.

\subsubsection{In three dimensions}

So here is the $v$ function, and what we need is an integration on
the atmosphere:
\begin{equation}
\intvol\left[\doverd{u}{t}v - \chi v + \diverge{\transport} v\right] \dd V = 0
\label{math:solve:3D_photo}
\end{equation}
with $V$ a volume element
\begin{equation}
  \intvol\doverd{u}{t}v\dd V 
- \intvol\chi v \dd V 
+ \intvol\diverge{\transport}v \dd V  = 0
\label{math:photo_3D_start}
\end{equation}
Thus, integration by parts will give,
using that $\left(\diverge{\transport}\right) v = \diverge{v\transport}$:
\begin{equation}
  \intvol\doverd{u}{t}v\dd V
- \intvol\chi v \dd V 
- \intvol \transport \diverge{v} \dd V 
+ \intboundary \vect{\transport}\vect{s}v \dboundary = 0
\label{math:photo_3D_fin}
\end{equation}
Using the formulation:
\begin{equation}
u = \sum_{i=1}^N \ui_i \phi_i,\qquad v = \sum_{i=1}^N\vi_i \phi_i
\end{equation}
with $\ui \in \mathcal{R}$ and true $\forall \vi \in \mathcal{R}$. 
we rewrite Eq.~\ref{math:photo_3D_fin}
\begin{equation}
\begin{split}
  \underbrace{\sum_{i=1}^N\sum_{j=1}^N\ui_i\vi_j\intvol\doverd{\phi_i}{t}\phi_j\dd V}_{\text{mass\_residual}}
- \underbrace{\sum_{j=1}^N\vi_j\intvol\chi \phi_j \dd V}_\text{element\_time\_derivative} \\
- \underbrace{\sum_{j=1}^N\vi_j\intvol \transport \diverge{\phi_j} \dd V}_\text{element\_time\_derivative}
+ \underbrace{\intboundary u\vect{\omega}\vect{s}v \dboundary}_\text{side\_time\_derivative} & = 0
\end{split}
\end{equation}

\subsubsection{In one dimension}

We can make the assumption that the only interesting dimension is the altitude, and
not caring for the other dimensions. Either everything
is considered independant or we really solve only a 1D problem, still,
the natural coordinate system is the spherical
coordinates system, thus we re-write Eq.~\ref{math:solve:3D_photo}, 
we integrate on $\theta$ and $\varphi$,
which gives a constant we can get rid of.
\begin{equation}
\intr\left[   \doverd{u}{t} v 
                       - \chi v + \frac{1}{r^2}\dOverd{r^2\transport}{r} v
                \right] \dVs = 0
\end{equation}
thus
\begin{equation}
\int_{r_0}^{r_1}\left[   \doverd{u}{t} v 
                       - \chi v + \doverd{\transport}{r} v
                       + \frac{2}{r}\transport v
                \right] r^2\dd r = 0
\end{equation}
now we integrate by parts:
\begin{equation}
\begin{split}
\int_{r_0}^{r_1}\doverd{\transport}{r} v r^2 \dd r 
   & = - \int_{r_0}^{r_1}\transport \left(\doverd{v}{r}r^2 + 2 r v\right) \dd r + \left[\transport v r^2\right]_{r_0}^{r_1}
        \\[\baselineskip]
   & = - \int_{r_0}^{r_1}\transport \doverd{v}{r} r^2  \dd r - 2 \int_{r_0}^{r_1}\transport  r v \dd r  + \left[\transport v r^2\right]_{r_0}^{r_1}
\end{split}
\end{equation}
So
\begin{equation}
\begin{split}
\int_{r_0}^{r_1} \left[   \doverd{u}{t} v 
                        - \chi v
                        + \frac{2}{r} \transport v
                        - u\omega\doverd{v}{r}
                        - \frac{2}{r} \transport v
                 \right] r^2 \dd r 
     + \left[\transport v r^2\right]_{r_0}^{r_1}   & = 0 \\
\int_{r_0}^{r_1} \left[   \doverd{u}{t} v 
                        - \chi v
                        - \transport\doverd{v}{r}
                 \right] r^2 \dd r 
     + \left[\transport v r^2\right]_{r_0}^{r_1}   & = 0 
\end{split}
\end{equation}
rewriting the transport term:
\begin{equation}
\int_{r_0}^{r_1} \left[   \doverd{u}{t} v 
                        - \chi v
                        - \left(\omega_A \doverd{u}{r} + \omega_B u \right)\doverd{v}{r}
                 \right] r^2 \dd r 
     + \left[(\omega_A \doverd{u}{r} + \omega_B u) v r^2\right]_{r_0}^{r_1}  = 0 
\end{equation}

using again the notations:
\begin{equation}
u = \sum_{i=0}^{N} \ui_i \phi_i, \qquad v = \sum_{i=0}^{N} \vi_i \phi_i
\end{equation}
We obtain:
\begin{equation}
\begin{split}
 \int_{r_0}^{r_1}\left[
                  \sum_{i=0}^{N}\ui_i\doverd{\phi_i}{t} \sum_{j=0}^{N}\vi_j \phi_j
                 - \chi \sum_{i=0}^{N}\vi_i\phi_i 
                 - \sum_{i=0}^{N}\ui_i\left(\omega_A\doverd{\phi_i}{r} + \omega_B \phi_i\right)
                     \sum_{j=0}^{N}\vi_j\doverd{\phi_j}{r}\right]r^2\dd r \\
 + \left[\sum_{i=0}^{N}\ui_i\left(\omega_A\doverd{\phi_i}{r} + \omega_B\phi_i \right)\sum_{j=0}^{N}\vi_j\phi_j r^2\right]_{r_0}^{r_1} = 0
\end{split}
\end{equation}
The rearrangement gives:
\begin{equation}
\begin{split}
  \underbrace{\sum_{i=1}^{N}\sum_{j=0}^{N}\ui_i\vi_j\int_{r_0}^{r_1}\doverd{\phi_i}{t}\phi_jr^2\dd r}_\text{mass\_residual}
- \underbrace{\sum_{i=0}^{N}\vi_i\int_{r_0}^{r_1}\chi\phi_i r^2\dd r}_\text{element\_time\_derivative} \\
- \underbrace{\sum_{i=0}^{N}\sum_{j=0}^{N}\ui_i\vi_j\int_{r_0}^{r_1}\omega_A\doverd{\phi_i}{r}\doverd{\phi_j}{r}r^2\dd r}_\text{element\_time\_derivative}
- \underbrace{\sum_{i=0}^{N}\sum_{j=0}^{N}\ui_i\vi_j\int_{r_0}^{r_1}\omega_B\phi_i\doverd{\phi_j}{r}r^2\dd r}_\text{element\_time\_derivative} \\
+ \underbrace{\sum_{i=0}^{N}\sum_{j=0}^{N}\ui_i\vi_j\left[\omega_A\doverd{\phi_i}{r}\phi_j r^2 + \omega_B\phi_i\phi_jr^2\right]_{r_0}^{r_1}}_\text{side\_time\_derivative}  & = 0
\end{split}
\end{equation}


At this point, we link to \GRINS\footnote{\GitGrins}: the planetary code provides $\chi$, $u$ and
$\omega$ for any $s$ species and $r$ \GRINS\ asks. \LibMesh\footnote{\GitLibmesh} is the underlying library that
does all the calculations.
