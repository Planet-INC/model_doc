\graphicspath{{./figs/}}
\makeatletter
%%% misc
\@addtoreset{chapter}{part}%because I want my chapter to be in parts
\newcommand{\mytilde}{\raisebox{-0.3\baselineskip}{\~{}}}
\newlength{\fakespace}
\newcommand{\fakeequalspace}{%
\settowidth{\fakespace}{= }%
\hspace{\fakespace}}
\newcommand{\mathdashrule}{%
\newcount\r%
\r=0
\loop\advance\r by1\rule{5mm}{0.4pt}\hspace{2.5mm}\ifnum\r < 2\repeat%
\rule{5mm}{0.4pt}\hspace{1.25mm} & 
\hspace{1.25mm}%
\loop\advance\r by1\rule{5mm}{0.4pt}\hspace{2.5mm}\ifnum\r < 4\repeat%
\rule{5mm}{0.4pt}%
}
%%% from report.cls, adapated
%%% added the makegraph stuff and
%% redefined \thesubsection to protect it and below
\newcommand{\makegraph}      {\relax}
\newcommand{\makeno@graph}   {\renewcommand{\makegraph}{\relax}}
\newlength{\graphsize}
\newcommand{\print@graph} [1]{{\includegraphics[width=0.9in]{#1}}}
\newcommand{\make@graph}  [1]{\renewcommand{\makegraph}{\noindent\make@@graph{\print@graph{#1}\rule{1ex}{0pt}}}}%
\newcommand{\make@@graph} [1]{\frame{\makebox[0pt][r]{\raisebox{-0.5\graphsize}[0pt][0pt]{#1}}}}%
\newcommand{\make@@@graph}[1]{\ifx#1\relax%
                               \makeno@graph%
                                \else%
                               \settoheight{\graphsize}{\print@graph{#1}}\make@graph{#1}%
                              \fi%
                            }%
\renewcommand\section[1][\relax]{\make@@@graph{#1}%
                                 \@startsection {section}{1}{\z@}%
                                 {-3.5ex \@plus -1ex \@minus -.2ex}%
                                 {2.3ex \@plus.2ex}%
                                 {\makegraph\normalfont\Large\bfseries}
                                }
%%%%%
%%
%%%
%%meta-meta-macro
\newcommand{\make@}[3][upshape]{%
\expandafter\gdef\csname font@#2\endcsname##1{% create font@something macro
{\csname#1\endcsname\csname#3\endcsname{##1}}%
}%
\expandafter\gdef\csname make@#2\endcsname##1{% create make@something meta macro
\expandafter\gdef\csname ##1\endcsname{\csname font@#2\endcsname{##1}}%
}
\expandafter\gdef\csname#2\endcsname##1{\csname font@#2\endcsname{##1}}%create the \something macro
}
%% libraries
\make@{library}{tt}
\newcommand{\lib@git} [3]{\href{https://github.com/#1/#2}{#3}}
%%
\make@library{Antioch} %%defines \Antioch
\make@library{GRINS}   %%defines \GRINS
\make@library{LibMesh} %%defines \LibMesh
\make@library{Eigen}   %%defines \Eigen
\newcommand{\PINC}{\library{Planet\_INC}}
\newcommand{\GitAntioch}{\lib@git{libantioch}{antioch}{\Antioch\ on Github}}
\newcommand{\GitGrins}  {\lib@git{GRINSfem}{GRINS}{\GRINS\ on Github}}
\newcommand{\GitLibmesh}{\lib@git{libMesh}{LibMesh}{\LibMesh\ on Github}}
\newcommand{\EigenTux}  {\href{eigen.tuxfamily.org/}{\Eigen\ on tuxfamily}}

%%% files
\newcommand{\file}[1]{\textsf{#1}}
\newcommand{\planetConstantsHeader}         {\file{\mytilde/planet/src/utilities/include/planet/planet\_constants.h}}
\newcommand{\antiochMathConstantsHeader}    {\file{\mytilde/antioch/src/utilities/include/planet/math\_constants.h}}
\newcommand{\antiochPhysicalConstantsHeader}{\file{\mytilde/antioch/src/utilities/include/planet/physical\_constants.h}}
\newcommand{\object@code} [1]{\texttt{#1}}

%%% versions
\newcounter{vmajor}
\newcounter{vmedium}
\newcounter{vminor}

\newcommand{\version}[3]{%
\setcounter{vmajor}{#1}
\setcounter{vmedium}{#2}
\setcounter{vminor}{#3}
}
%
\newcommand{\theversion}{\font@library{\thevmajor.\thevmedium.\thevminor}}

%%% math
\newcommand{\doverd}        [2]{\ensuremath{\frac{\partial#1}{\partial#2}}}
\newcommand{\ddoverd}       [2]{\ensuremath{\frac{\partial^2#1}{\partial{#2}^2}}}
\newcommand{\dOverd}        [2]{\doverd{\left(#1\right)}{#2}}
\newcommand{\doverdz}       [1]{\doverd{#1}{z}}
\newcommand{\ddoverdz}      [1]{\ddoverd{#1}{z}}
\newcommand{\dOverdz}       [1]{\dOverd{#1}{z}}
\newcommand{\doverdt}       [1]{\ensuremath{\frac{\dd #1}{\dd t}}}
\newcommand{\vect}          [1]{\ensuremath{\overrightarrow{#1}}}
\newcommand{\diverge}       [1]{\ensuremath{\vect{\nabla}\cdot(\vect{#1})}}
\newcommand{\divergex}      [2]{\ensuremath{\partial_{#1}{\vect{#2}}}}
\newcommand{\divergez}      [1]{\divergex{z}{#1}}
\newcommand{\gradien}       [2]{\ensuremath{\text{grad}_{#1}\vect{#2}}}
\newcommand{\gradienz}      [1]{\gradien{z}{#1}}
\newcommand{\vectProd}      [2]{\ensuremath{\vect{#1}\cdot\vect{#2}}}
\newcommand{\vectprod}      [2]{\ensuremath{#1\cdot#2}}
\newcommand{\coeffderiv}    [3]{\ensuremath{a^{(#1)}_{#2#3}}}
\newcommand{\coeffderivdiff}[2]{\coeffderiv{\conc_s\omega_s}{#1}{#2}}
\newcommand{\mat}           [1]{\ensuremath{\left[\bf #1 \right]}}
%%% global
\newcommand{\h}        {\ensuremath{\mathrm{h}}}
\newcommand{\dd}       {\ensuremath{\mathrm{d}}}
\newcommand{\chapman}  {\ensuremath{\text{Chap}(\x,\chi)}}
\newcommand{\erf}      {\ensuremath{\text{erf}}}
\newcommand{\kb}       {\ensuremath{\text{k}_\text{b}}}
\newcommand{\Nav}      {\ensuremath{\mathcal{N}_\text{Avo}}}
\newcommand{\RTitan}   {\ensuremath{\text{R}_\text{Titan}}}
\newcommand{\MTitan}   {\ensuremath{\text{m}_\text{Titan}}}
\newcommand{\Guni}     {\ensuremath{\mathcal{G}}}
\newcommand{\gloc}     {\ensuremath{g}}
\newcommand{\x}        {\ensuremath{a}}
\newcommand{\sch}      {\ensuremath{H}}
\newcommand{\mfp}      {\ensuremath{l}}
\newcommand{\RPlanet}  {\ensuremath{\text{R}_\text{Planet}}}
\newcommand{\MPlanet}  {\ensuremath{\text{m}_\text{Planet}}}
%%%%% hv
\newcommand{\flux}     {\ensuremath{\Phi}}
\newcommand{\fhv}      {\ensuremath{\flux^{\h\nu}}}
\newcommand{\fhvAU}    {\ensuremath{\fhv_\text{1\,AU}}}
\newcommand{\fhvtop}   {\ensuremath{\fhv_\infty}}
\newcommand{\fe}       {\ensuremath{\flux^{\ce{e-}}}}
\newcommand{\dSS}      {\ensuremath{\text{d}_\text{Sun-Saturn}}}
\newcommand{\cs}       {\ensuremath{\sigma}}
%%%%% chemistry
\newcommand{\conc}    {\ensuremath{n}}
\newcommand{\concAtm} {\ensuremath{n_\text{total}}}
\newcommand{\mass}    {\ensuremath{m}}
\newcommand{\Mm}      {\ensuremath{\text{M}}}
%%%%
\newcommand{\zmin}    {\ensuremath{z_\text{min}}}
\newcommand{\mean} [1]{\ensuremath{\langle#1\rangle}}
\newcommand{\unit} [1]{\ensuremath{\mathtt{#1}}}
\newcommand{\uu}      {\ensuremath{\tt}}
\newcommand{\nounit}  {\unit{no\ unit}}
\newcommand{\ukb}     {\unit{m^2\,kg\,s^{-2}\,K^{-1}}}
\newcommand{\uGuni}   {\unit{m^3\,kg^{-1}\,s^{-2}}}
%% diffusion
\newcommand{\diff}    {\ensuremath{D}}
\newcommand{\eddy}    {\ensuremath{K}}
\newcommand{\Tz}      {\ensuremath{\mathrm{T_0}}}
\newcommand{\Pz}      {\ensuremath{\mathrm{P_0}}}
\newcommand{\Amas} [1]{\ensuremath{\diff_{#1}(0,1)}}
\newcommand{\smas} [1]{\ensuremath{\beta_{#1}}}
\newcommand{\diff@}[2]{\ensuremath{#1^{\text{(#2)}}}}
\newcommand{\Pwa}     {\diff@{\mathrm{P}}{wa}}
\newcommand{\Awa}  [1]{\diff@{A_{#1}}{wa}}
\newcommand{\swa}  [1]{\diff@{\beta_{#1}}{wa}}
\newcommand{\concwa}  {\diff@{\conc}{wa}}
\newcommand{\Awi}  [1]{\diff@{A_{#1}}{wi}}
\newcommand{\swi}  [1]{\diff@{\beta_{#1}}{wi}}
\newcommand{\Jeans}   {\ensuremath{J}}
%% molecules
\newcommand{\thermcoef}   {\ensuremath{\alpha}}
\newcommand{\conductivity}{\ensuremath{\kappa}}
\newcommand{\viscosity}   {\ensuremath{\nu}}
\newcommand{\visC}        {\ensuremath{C}}
\newcommand{\molarfrac}   {\ensuremath{x}}
\newcommand{\specHeat}    {\ensuremath{{C_p}}}
\newcommand{\HeatRate}    {\ensuremath{H}}
\newcommand{\heat}        {\ensuremath{Q}}
\newcommand{\heatEff}     {\ensuremath{\epsilon}}
%%% kinetics
\newcommand{\Tref}    {\ensuremath{\mathrm{T_{ref}}}}
\newcommand{\preCoeff}{\ensuremath{A}}
\newcommand{\powerPar}{\ensuremath{\beta}}
\newcommand{\actEn}   {\ensuremath{E_a}}
\newcommand{\Runi}    {\ensuremath{\mathrm{R}}}
\newcommand{\order}[1]{\ensuremath{{m_{#1}}}}
\newcommand{\stoi} [1]{\ensuremath{{\nu_{#1}}}}
\newcommand{\rate}    {\ensuremath{k}}
%%%% solving
\newcommand{\omegadot}   {\ensuremath{\dot{\conc}^{(\chi)}}}
\newcommand{\Rbot}       {\ensuremath{\mathrm{r_{\text{bottom}}}}}
\newcommand{\Rtop}       {\ensuremath{\mathrm{r_{\text{top}}}}}
\newcommand{\Nd}         {\ensuremath{\mathrm{N_d}}}
\newcommand{\coeftype}[1]{\ensuremath{\mathtt{#1}}}
%\newcommand{\coefconc}[1]{\ensuremath{\coeftype{\conc}_{#1}}}
%\newcommand{\coefv}   [1]{\ensuremath{\coeftype{c}_{#1}}}
%%%% math solve
\newcommand{\ui}        {\ensuremath{\coeftype{u}}}
\newcommand{\vi}        {\ensuremath{\coeftype{v}}}
\newcommand{\matrice}[1]{\ensuremath{\mathbf{#1}}}
%%%%%%%
% remarks
%%%%%%%
\newenvironment{remark}
{%
\renewcommand{\labelitemi}{\textemdash}
\addtolength{\leftmargin}{15pt}
\par
\noindent\rule{\linewidth}{0.4pt}\\*%
\underline{\bf Remark}:%
\begin{adjustwidth}{12pt}{}
\small\sf%
}
{\end{adjustwidth}
\noindent\rule{\linewidth}{0.4pt}}
%%%%%%%
% chemical equation
%%%%%%%
\newcounter{chemReac}
\renewcommand{\thechemReac}{$\chi$\;\thechapter.\arabic{chemReac}}
\newenvironment{chemequation}
{\begin{displaymath}
\refstepcounter{chemReac}\tag{\thechemReac}}
{\end{displaymath}}
%%%%%%%
% equation with ref to object
%%%%%%%
%% first, renew the tag
\newlength{\EqCodew}
\newcommand{\print@Object}{\relax}
\newenvironment{equationCode}[1]
{%
\renewcommand{\theequation}{\arabic{chapter}.\arabic{equation}\protect\print@Object}
\settowidth{\EqCodew}{\object@code{#1}}%
\ifdim\EqCodew>1.2in
  \setlength{\EqCodew}{1.2in}%
\fi
\renewcommand{\print@Object}{\makebox[0pt][l]{\rule{1.1ex}{0pt}\resizebox{\EqCodew}{!}{\object@code{#1}}}}
\begin{equation}
}
{\end{equation}}
%%%%%%%
% notations
%%%%%%%
\newcounter{notationCounter}
\renewcommand{\thenotationCounter}{Notation~\arabic{notationCounter}}
\newenvironment{notation}
{\begin{displaymath}
\refstepcounter{notationCounter}\tag{\thenotationCounter}}
{\end{displaymath}}
%%%%%%%
% warning
%%%%%%%
\newcommand{\warningsign}{%
\begin{tikzpicture}[baseline = (x.base)]
\node[regular polygon, regular polygon sides=3,draw=white,double=red,
      fill=red,text=white,font=\bf\large,rounded corners] (x) {!};
\end{tikzpicture}
}
\newcommand{\warning}[1]{%
\hspace{-1in}\warningsign\textbf{\textcolor{red}%
{\large\sc #1}}
}
\makeatother
