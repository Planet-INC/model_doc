\begin{figure}
\centering
\includegraphics{atm_layers}
\caption{\label{T(z):Titan}Temperature profile of Titan's atmosphere}
\end{figure}

\warning{Not implemented yet}

The heat transfer equation is:
\begin{equation}
\concAtm(z)\specHeat_{\text{atm}}(z)\doverd{T}{t} 
                                     = \frac{1}{\left(\RTitan + z\right)^2} 
                                       \doverd{\left(\left(\RTitan + z\right)^2
                                               \conductivity_\text{atm}(z)\doverd{T}{z}\right)
                                              }{z}
                                        + \HeatRate_{\text{\sc heat}}
                                        + \HeatRate_{\text{\sc cooling}}
\label{Titan:temp}
\end{equation}
with the atmospheric specific heat $\specHeat_\text{atm}$ (\unit{J\,mol^{-1}\,K^{-1}} or \unit{J\,kg^{-1}\,K^{-1}}) and
$\conductivity_\text{atm}$ the atmospheric thermal conductivity (\unit{W\,m^{-1}\,K^{-1}}),
$\HeatRate_{\text{\sc heat}}$ the heating rate (\unit{J\,m^{-3}\,s^{-1}}) and
$\HeatRate_{\text{\sc cooling}}$ the cooling rate (\unit{J\,m^{-3}\,s^{-1}}).

\section{Atmospheric specific heat}
It is defined as:
\begin{equation}
\specHeat_{\text{atm}}(z) = \sum_s \molarfrac_s(z)\specHeat_s(z)
\label{Titan:spec_heat}
\end{equation}

\section{Atmospheric conductivity}

\begin{equation}
\conductivity_\text{atm}(T) = \sum_{s \in \{\ce{N2}, \ce{CH4}\}} 
                              \frac{\conductivity_s(T,z)}
                                   {\displaystyle\sum_{j\neq s}\frac{\numprint{1.065}}{2\sqrt{2\left(1+\frac{\mass_s}{\mass_j}\right)}}
                                    \left[1 + \sqrt{\frac{\viscosity_s(T)}{\viscosity_j(T)}\sqrt{\frac{\mass_j}{\mass_s}}}\right]^2
                                    \frac{\molarfrac_j(z)}{\molarfrac_s(z)}
                                   }
\label{Titan:atmcond}
\end{equation}
Approximating Titan's
atmosphere by a mixture of \{\ce{N2},\ce{CH4}\}, we have as conductivity 
$\conductivity_i$ of \ce{N2} and \ce{CH4},
\begin{equation}
\begin{split}
\conductivity_{\ce{N2}}(T)  & = \numprint{0.00309} + \numprint{7.593}\,10^{-5}\, T - \numprint{1.1014}\,10^{-8}\, T^2 \\
\conductivity_{\ce{CH4}}(T) & = \numprint{-0.00935} + \numprint{1.4028}\,10^{-4}\, T + \numprint{3.318}\,10^{-8}\, T^2
\end{split}
\label{Titan:conduc}
\end{equation}
and viscosities $\viscosity_i$ (\unit{P}--poise):
\begin{equation}
\begin{split}
\viscosity_{\ce{N2}}(T)  & = \viscosity_{0,\ce{N2}}\frac{T_{0,\ce{N2}} + \visC_{\ce{N2}}}{T + \visC_{\ce{N2}}}\left(\frac{T}{T_{0,\ce{N2}}}\right)^{\frac{3}{2}} \\
\viscosity_{\ce{CH4}}(T) & = \left(\numprint{3.8435} + \numprint{0.40112}\, T - \numprint{1.4303}\,10^{-4}\, T^2\right)\,10^{-4}
\end{split}
\label{Titan:visco}
\end{equation}
with \visC\ the Sutherland constant (\unit{K}).

\section{Heating rate}

\subsection{Solar heating rate}

The solar heating rate is given by equation~\ref{Titan:solar-heat-rate}
\begin{equation}
\HeatRate_{\text{solar,atm}}(z,\lambda) = \heatEff^{\text{solar}}(z,\lambda)\heat_{\text{solar,atm}}(z,\lambda)
\label{Titan:solar-heat-rate}
\end{equation}
Using the solar heating efficiency of the atmosphere $\heatEff^{\text{solar}}$ and
the absorbed solar energy $\heat_{\text{solar,atm}}$:
\begin{equation}
\heat_{\text{solar,atm}}(z,\lambda) = \fhv(z,\lambda)\sum_s \cs_s(\lambda)\conc_s(z)
\label{Titan:solar-heat}
\end{equation}

\subsection{Magnetospheric heating}

Heating du to electrons flux through the atmosphere, same than
photon, only it's electrons.
\begin{equation}
\HeatRate_{\text{elec,atm}}(z,E) = \heatEff^{\text{elec}}(z,E)\heat_{\text{elec,atm}}(z,E)
\label{Titan:elec-heat-rate}
\end{equation}
Using the electronic heating efficiency of the atmosphere $\heatEff^{\text{elec}}$ and
the absorbed electronic energy $\heat_{\text{elec,atm}}$:
\begin{equation}
\heat_{\text{elec,atm}}(z,E) = \fe(z,E)\sum_s \cs^{\text{elec}}_s(E)\conc_s(z) \times 40~\unit{eV}
\label{Titan:elec-heat}
\end{equation}
with the factor 40~\unit{eV}, the energy needed on average to break an
electron-ion pair.

\subsection{Global heating rate}

Is defined a global heating rate profile:
\begin{equation}
\heatEff_\text{g} = \frac{\displaystyle\sum_\lambda\HeatRate_{\text{solar,atm}}(z,\lambda) + 
                          \displaystyle\sum_E\HeatRate_{elec,atm}(z,E)}
                         {\displaystyle\sum_\lambda\heat_{\text{solar,atm}}(z,\lambda) + 
                          \displaystyle\sum_E\heat_{elec,atm}(z,E)}
\label{Titan:global-heat}
\end{equation}

\section{\texorpdfstring{\ce{HCN} rotational cooling}{Hydrogen cyanide rotational cooling}}
